\section{吉布斯采样}
\label{sec:17.4}

	到目前为止,本章已经介绍了如何通过迭代更新$x\rightarrow x' \sim T (x'|x) $在概率分布$q(x)$上采样的方法.
但是,还没有说明怎样确保$q(x)$是有效的概率分布.
本书主要考虑两种基本方法.第一种方法根据给定的通过学习得到的$p_{model}$推导出$T$,下文将会详细介绍从EBMs(energy-based model)采样的例子.
第二种方法是直接参数化$T$并学习,使其平稳分布可以隐式地定义$p_{model}$的兴趣点.
第二种方法的例子将在章节"\textcolor{red}{20.12}"和"\textcolor{red}{20.13}"阐述.

在深度学习中,通常使用马尔科夫链从以基于能量的模型定义的概率分布$p_{model}$中采样.
此例中,马尔科夫链所需的$q(x)$就是$p_{model}$.
为得到满足需求的$q(x)$,必须选择合适的$T(x'|x)$.

为了构建马尔科夫链,一种概念上简单有效的方法就是使用\textit{吉布斯采样}从$p_{model}$中采样.
在吉布斯采样中,通过选择一个变量$x_i$实现从$T(x'|x)$中采样,$x_i$的选择方法取决于它在基于能量模型结构的无向图$\mathcal{G}$上的邻居变量.
只要给定的所有邻居节点都条件独立,也可以同时对多个变量采样.
在16.7.1中的RBM(受限玻尔兹曼机)例子中,RBM的所有隐含层单元可以被同时采样,是因为他们都条件独立于其他的给定可视层单元.
同样的,因为对于给定的隐含层单元,所有的可视层单元都是条件独立的,因此所有的可视层单元也可以同时被采样.
以这种方式同时更新多个变量的吉布斯采样方法,被称为块吉布斯(block gibbs)采样.

从$p_{model}$中采样来设计马尔科夫链的代替方法是可行的.例如,Metropolis-Hastings算法就广泛应用于其他学科领域.
在深度学习领域实现无向建模,除吉布斯采样外很少使用其他方法.
改进采样技术是一个潜在的研究领域.